\documentclass[/../base.tex]{subfiles}

\begin{document}
\section{Data}
\label{data}

The data were collected from Datastream on July 2nd 2015. The countries under consideration - France, Greece, Ireland, Italy, Portugal, Spain, and the Netherlands, with Germany included for reference - comprise three `core' and five `periphery' Eurozone states. This allows comparative analysis between sovereigns in different regions. Though the exclusion of smaller Euro-area countries such as Slovakia or Finland may induce sample selection bias, the eight nations listed above were the only single currency members for whom daily data are available over the entire sample period, for all variables. 

The variable of interest in all specifications is the end-of-day yield on a ten-year, zero-coupon bond, taken as a deviation from the German equivalent. The results of an Augmented Dickey-Fuller (ADF) test are reported in Table \ref{tab:adf}. The null of a unit root cannot be rejected at any reasonable confidence level for any country, with the exception of the Netherlands at 10\%. Differencing the series results in comfortable rejection of the null, and this forms the final dependent variable.  

\begin{table}[h]
	\centering
	\begin{threeparttable}
		\caption{Augmented Dickey Fuller Statistics}
		\begin{tabular}{l c*{3}{c}}
			\toprule
			\toprule
			Country & bsp & $\Delta$bsp\\
			\midrule
			Spain     &   -0.8594 &   -33.1702  \\
			France  &   -1.2618 &   -32.3315\\
			Greece &   -0.6546 &   -33.1763  \\
			Ireland   &   -0.8241 &   -29.1334 \\
			Italy     &   -0.8146 &   -35.2573 \\
			Portugal  &   -0.7628 &   -30.0825\\
			Netherlands &   -1.3654 &   -33.7947  \\
			\bottomrule
			\bottomrule
		\end{tabular}
		\centering
		\begin{tablenotes}
			\small
			\item Notes: Critical values: -1.2816 (10\%), -1.6449 (5\%), -2.3263 (1\%) H$_0$: At least one unit root.
		\end{tablenotes}
		\label{tab:adf}
	\end{threeparttable}
\end{table}

A plot of the differenced bond spreads for each country, shown in Figure \ref{fig:dbsp}, clearly demonstrates volatility clustering. Initially, this is helpful for generating crisis dates, where the indicator function is triggered when spreads exceed a given multiple of the standard deviation. Once the crisis dates are generated, however, volatility persistence could have implications for analysing contagion. Firstly, as with all   heteroskedasticity that is unaccounted for, linear estimators will be less efficient. Standard errors will be underestimated and the null rejected too frequently. Importantly for the analysis presented here, neglecting volatility is likely to lead to more frequent rejection of the null of no contagion. As a result, a heteroskedastic and autocorrelation (HAC) robust covariance estimator is used. 

\begin{figure}
	\centering
	\includegraphics[width = \textwidth]{../../Data/graphics/d_bsp_graph.pdf}
	\caption{Differenced bond spreads for sample countries.}
	\label{fig:dbsp}
\end{figure}

Many recent analyses of market reactions to sovereign default risk have used credit default swap (CDS) spreads as a dependent variable. It is important to note that, while ostensibly determined by a similar underlying process, the evolution of CDS premia and bond yields is not identical. As outlined in  \cite{fontana2010analysis} and \cite{beirne2013pricing}, the former measure suffers from several complications relating to investor risk-appetite and market liquidity that make it less suitable for drawing policy related conclusions.

Common factor variables are included to account for any Europe-wide shocks. This is especially important when examining systemic crises, where overall market risk appetite or attitude to uncertainty may change in response. Following \cite{metiu2012sovereign}, the lagged spread between the Euro Interbank Offered Rate (Euribor) and German Treasury bills is used as a general European risk premium, and the log-differenced VSTOXX index is interpreted as the change in market-expected volatility. \cite{giordano2013pure} use the VIX index as an alternative to the latter, though this is based on U.S. stock volatility and so is better seen as a measure of global risk conditions.  

As noted in Section \ref{est}, identification of the contagion coefficient under a GIVE setup requires the presence of country specific factors to use as instruments for crisis dummies in the other country equations. A number of candidates could be suggested, however an intransigent issue in estimating linkage models is the low-frequency of macro data. Daily observations of the dependent variable are required to evade the endogeneity problems discussed in Section \ref{dating_methodology}. Even weekly or monthly observations are almost certainly too infrequent to capture the true response to a shock. With variables such as output and employment available only in monthly or quarterly varieties, it is hard to justify their use, despite their intuitive appeal. 

Excluding these indicators, however, may introduce new problems. It may be the case that the country-specific variables available at a daily frequency are insufficient to identify the contagion coefficient, as discussed in Section \ref{dating_methodology}. In this case, again following \cite{metiu2012sovereign}, a Euro denominated daily domestic S\&P stock index for each country is used. There are obvious endogeneity concerns, as it is highly likely that equity prices are determined at least partially by the common factors. However, likely due to lack of available alternatives, these indices have become a standard tool in the literature.


\end{document}
\documentclass[../base.tex]{subfiles}

\begin{document}
\section{Estimation Methodology}
\label{est}

There is substantial econometric literature on the estimation of linkage models. 

Recap: `monsoonal' effects are underlying variables are correlated, `spillovers' are through trade and financial linkages, 

Pure correlation tests are not valid in presence of interdependencies. See Pesaran and Pick for monte carlo evidence for bias in presence. 

General problem of inference in non-linear simultaneous equation models. 

To identify contagion coefficient, need to have market specific variables included in model OR assume no interdependencies.

If do not have specific variable to identify then must assume that crisis periods can be identified a-priori, and that there are sufficient crisis periods for consistent estimation of the correlation coefficients under both regimes. Parameter stability tests can then be applied. 

PP estimate a bias of ... and propose a framework for 



See multi-eqn PP section on how crisis indicator works there

In this study, three methods of crisis period delimitation are used. First, in the most simple specification, all periods following the Greek debt announcement are determined as an a-priori specification of crisis. I.e. $D_{i,t} = 1$ for $t > 2009-10-20$, and zero otherwise. Second, following (), a stress period period is defined as bond spread exceeding a threshold of one and a half standard deviations. Formally: $D_{i,t} = I(1.5\sigma_{j,t}), \forall i, j \in \{1, N\}$. Finally, as in \cite{metiu2012sovereign}, a Value at Risk (VaR) model is used. 



KEY POINTS

1) Identification of crisis dates.
2) Dealing with heteroskedasticity and other basic econometric problems. 
3) Other forms of endogeneity. 


Terminology list:

$s_{it}$: bond yield/CDS spread (dependent variable)

$\alpha_j$: Fixed effect coefficient

$\beta$: Contagion coefficient (multiple possible transmission paths)

$\delta$: Common factors coefficients, as in PP

$\gamma$: Idiosyncratic factors coefficients ($\alpha$ in PP)

$D_t$: Crisis dummy (defined in PP model as $D_{it} = I(s_{jt} - c_j\sigma{j, t-1}$)

$X_{it}$: Idiosyncratic variables

$Z_t$: Common factors. 

\subsection{Na\"{i}ve Estimation}

\cite{giordano2013pure} describe two empirical models to provide an initial estimate of contagion coefficients. 

Stationary case:

\begin{align}
	s_{it} = \alpha_{i0} + \alpha_{1}s_{it-1} +\gamma X_{it} + \delta Z_t + \beta_0D_t + \beta_1D_ts_{it-1} + \beta_2D_tX_{it} + \beta_3D_tZ_t + \varepsilon_{it}
\end{align}

Non-stationary:

\begin{align}
	s_{it} = \alpha_{i0} + \gamma X_{it} + \delta Z_t + \beta_0D_t + \beta_2D_tX_{it} + \beta_3D_tZ_t + \varepsilon_{it}
\end{align}


\subsection{Pesaran and Pick Estimation}
\label{pp}

\begin{align}
	s_{it} =&~\gamma_i X_{it} + \delta Z_t + \beta_0D_t + \beta_2D_tX_{it} + \beta_3D_tZ_t + \varepsilon_{it} \\
	s_{it} =&~\alpha_{i0} + \gamma X_{it} + \delta Z_t + \beta_0D_t + \beta_2D_tX_{it} + \beta_3D_tZ_t + \varepsilon_{it} 
\end{align}



\end{document}
\documentclass[../base.tex]{subfiles}

\begin{document}
\section{Estimating Contagion}
\label{est}

\cite{pesaran2007econometric} (PP) consider a two country framework to capture the insights of \cite{masson1999contagion}. The model can be represented by the relations:

\begin{align} 
	s_{1,t} =&~\boldsymbol{\delta}_1^{\prime} \mathbf{x}_{1,t} + \boldsymbol{\phi}_1^{\prime} \mathbf{F}_t + \beta_1 D_{2,t} + u_{1,t} \label{eqn:system}\\
	s_{2,t} =&~\boldsymbol{\delta}_2^{\prime} \mathbf{x}_{2,t} + \boldsymbol{\phi}_2^{\prime} \mathbf{F_t} + \beta_2 D_{1,t} + u_{2,t} \label{eqn:system2}
\end{align}

where $s_{i,t}$ is the spread of country $i$ bond yields over the German equivalent, $\mathbf{x}_{i,t}$ is a vector of country-specific regressors (including any lags of the dependent variable), $\mathbf{F}_t$ is a vector of factors common to all countries, and $D_{j, t}$ is a dummy variable indicating a crisis in country $j \neq i$ at time $t$. $u_{i, t}$ is a serially uncorrelated error term with idiosyncratic conditional variances $\sigma^2_{ui, t-1}, i = 1, 2$ and a non-zero correlation coefficient $\rho$ that is assumed to be time invariant for ease of exposition. 

The external crisis variable $D_{j, t}$ takes a value of either zero or unity, and the formulation of this variable is discussed in detail in Section \ref{dating_methodology}. For now, it is sufficient to assume an indicator function is triggered if country $j$ sees spreads rise above a certain threshold $c_j$. Essentially, $D_{j,t}$ is an endogenous variable of the form $D_{j, t} = \mathbf{I}(s_{j, t} - c_j)$ where $\mathbf{I}(A) = 1$ if $A >0$ and zero otherwise.

After some simplifications, PP demonstrate that the analytical solution of the model depends crucially upon the strength of fundamentals - defined as $m_{i,t} = \boldsymbol{\delta}_i^{\prime} \mathbf{x}_{i,t} + \boldsymbol{\phi}_i^{\prime} \mathbf{F}_t + u_{i,t}$ - and the characteristics of the contagion coefficients. 

If one of $\beta_1$ or $\beta_2$ is zero, then the model possesses a simple solution. Abstracting from the possibility of time-varying thresholds, the system becomes

\begin{align*}
	s_{1,t} =&~ m_{1,t} + \beta_1 I(s_{2,t} - c_2) \\
	s_{2,t} =&~ m_{2,t}
\end{align*}

when $\beta_2 = 0$. $s_{2,t}$ is determined solely by its own fundamentals, while $s_{1,t}$ is a piecewise function of the realised $s_{2,t}$:
\[
s_{1,t} = 
\begin{cases} 
m_{1,t} + \beta_1  & \text{if } s_{2,t} > 0 \\
m_{1,t}		       & \text{if } s_{2,t} \leq 0
\end{cases}
\]

With two positive contagion coefficients, the solution is not unique for certain values of the fundamental variables. Normalising by the threshold and contagion coefficients, such that normalised variable $A_{i,t} =~\frac{a_{i,t} - c_i}{\beta_i}$:

\begin{align*}
	S_{1,t} =&~ M_{1,t} + I(S_{2,t})\\
	S_{2,t} =&~ M_{2,t} + I(S_{1,t})
\end{align*}

which gives a reduced formulation for country 1:

\begin{align}
	S_{1,t} =&~ M_{1,t} + I(M_{2,t} + I(S_{1,t}))
	\label{eqn:model_soln}
\end{align}

Given the dichotomous nature of the nested indicator function, there are four possible regimes for the system at time $t$:

\begin{align}
	\label{eqn:regions}
	\text{regime 1:}~ s_{1,t} & \leq	c_1, s_{2,t} \leq c_2; ~~~~ \text{regime 2:}~ s_{1,t} \leq	c_1, s_{2,t} > c_2 \\
	\text{regime 3:}~ s_{1,t} & >	c_1, s_{2,t} \leq c_2; ~~~~ \text{regime 4:}~ s_{1,t} >	c_1, s_{2,t} > c_2
\end{align}

These regimes can help to reach a solution for equation (\ref{eqn:model_soln}). Normalisation means that:

\begin{align*}
	\text{regime 1:}~ S_{1,t} & \leq	0, S_{2,t} \leq 0; ~~~~ \text{regime 2:}~ S_{1,t} \leq	0, S_{2,t} > 0 \\
	\text{regime 3:}~ S_{1,t} & >	0, S_{2,t} \leq 0; ~~~~ \text{regime 4:}~ S_{1,t} >	0, S_{2,t} > 0
\end{align*}

When the realisations of $S_{i,t}, i = 1,2$ are known then we can find the characteristics of the fundamentals. For example, in regime 2, $S_{1,t} = M_{1,t} + 1, M_{1,t} \leq 1; S_{2,t} = M_{2,t} > 0$. The full set of regimes is, in $(M_{1,t}, M_{2,t})$ space:

\begin{align*}
	\text{regime 1:}~ M_{1,t} & \leq	0, M_{2,t} \leq 0; ~~~~ \text{regime 2:}~ M_{1,t} \leq	-1, M_{2,t} > 0 \\
	\text{regime 3:}~ M_{1,t} & >	0, M_{2,t} \leq -1; ~~~~ \text{regime 4:}~ M_{1,t} >	-1, M_{2,t} > -1
\end{align*}

A fifth region, where $-1 < M_{1,t} \leq 0; -1 < M_{2,t} \leq 0$, where fundamentals in both countries are weak, but not so much so that there is no chance of the good outcome, does not directly correspond to a region in $(S_{1,t}, S_{2,t})$ space. In this region of fundamentals, $S_{i,t}$ is randomised between `good' and `bad' states according to some process. PP provide the example of $\mathbf{M}_t = (-\frac{1}{2}, -\frac{1}{3})^\prime$, for which the solutions $\mathbf{S}_t^a = (-\frac{1}{2}, -\frac{1}{3})^\prime$ and $\mathbf{S}_t^b = (\frac{1}{2}, \frac{2}{3})^\prime$ can both hold.

Designating the equilibrium choice by $d_t \sim$ Bernoulli($\pi^E$), where $\pi^E$ is the probability of the `bad' outcome of higher spreads, the solution in this region is characterised as $S_{i,t}^*(d_t) =~d_tM_{i,t} + (1 - d_t)(1 + M_{i,t})$. This generates a bimodal distribution of $s_{i,t}^* = s_{i,t}(d_t)$ that Pesaran and Pick demonstrate becomes more pronounced with larger contagion coefficients. Another feature of the model made clear by the above solution is that the $s_{1,t}$ and $s_{2,t}$ will be correlated through the crisis dummy even when fundamentals are uncorrelated, provided $\beta_i \geq 0, \beta_{j \neq i} > 0;~~ i,j = 1, 2$. 

The solution derived in the two-country setting is extended to the N-country formulation:

\begin{align}
	s_{i,t} =&~\boldsymbol{\delta}_1^{\prime} \mathbf{x}_{i,t} + \boldsymbol{\phi}_i^{\prime} \mathbf{F}_t + \sum_{j=1}^{N}\beta_j D_{j,t} + u_{i,t}, ~~~ i = 1,2,...,N ; j \neq i
	\label{eqn:pp_multi}	
\end{align}

and his equation will form the foundations of the estimation. The remainder of this section describes the econometric issues with estimating the contagion coefficients, as well as a summary of the techniques used. 

Finally, while not a primary concern of this paper, it is worth noting that the structure of the dataset has implications for identification. In the case of a sufficiently large number of countries, i.e. $N \rightarrow \infty$, the country-specific contagion coefficients cannot be identified, but for fixed $N$, $T \rightarrow \infty$, the multi-country model converges to the solution of the two-country system. As the analysis is limited to the Eurozone over several years of daily observations, it is reasonable to assume the latter setting applies. The formal result, derived in section 4 of \cite{pesaran2007econometric}, is therefore not reproduced.


\subsection{Estimation Strategies}
\label{est_strat}

In this study, three estimators will be described. The first two, OLS and Generalised Instrumental Variables, are used to estimate the model, while a Maximum Likelihood estimator is derived in the final subsection, although it is not utilised ue to complexities in implementing it for this set up. Again, for ease of notation, the two-country model will be used to describe the estimators. For the first two methods, they are subsequently extended to the N country case using equation (\ref{eqn:pp_multi}).

\subsubsection{OLS and GIVE}

Using the standard formulations in \cite{hayashi2000econometrics} as a foundation, and the expositions found in \cite{pesaran2007econometric} and \cite{massacci2007identification}, the OLS and Generalised Instrumental Variables (GIVE) estimator are derived below.

If we treat the dates of the crisis as predetermined, the dummy variables $D_{j,t}, ~ j = 1, 2$ are exogenous and accurately capture the two regimes. OLS provides an unbiased and consistent estimator in this setting. However, as discussed below, in section \ref{dating_methodology}, these assumptions are unlikely to hold, and $D_{j,t}$ will be endogenously determined. 

\cite{pesaran2007econometric} and \cite{metiu2012sovereign} use Generalised Instrumental Variable Estimation (GIVE) to consistently estimate the contagion coefficient. It is assumed that $D_{j,t},~ j = 1, 2$ is endogenous, while all other variables are predetermined. It is also assumed that the threshold parameters $c_1$ and $c_2$ are known. Vectors $\boldsymbol{\gamma}_i$, $\mathbf{y}_i$, and $\mathbf{h}_{i,t}$, for $i,j = 1,2 ~~ i \neq j$ are defined as 

\begin{align*}
 \boldsymbol{\gamma}_i \equiv&~ (\boldsymbol{\phi}_i^{\prime}, \boldsymbol{\delta}_{i}^{\prime}, \beta_i)^{\prime} \\
 \mathbf{y}_i \equiv&~ (y_{i,1},...,y_{i,T})^{\prime}\\
 \mathbf{h}_{i,t} \equiv&~ [\mathbf{F}_t^{\prime}, \mathbf{x}_{i,t}^{\prime}, \mathbf{I}(y_{j,t} - c_j)]^{\prime}
\end{align*}

with the matrix $\mathbf{H}_i \equiv (\mathbf{h}_{i,1}^{\prime},..., \mathbf{h}_{i,T}^{\prime})^{\prime}$ representing the exogenous and endogenous regressors for country $i$ across the whole sample period $T$. 

Defining $\mathbf{w}_{i,t}$ as the vector of instruments, $\mathbf{W}_i \equiv (\mathbf{w}_{i,1}^{\prime},..., \mathbf{w}_{i,T}^{\prime})^{\prime}$ is the full sample matrix of instruments, and with associated projection matrix: $\mathbf{P}_{\mathbf{W}_i} \equiv \mathbf{W}_i (\mathbf{W}_i^{\prime} \mathbf{W}_i)^{-1} \mathbf{W}_i^{\prime}$. The system in equations (\ref{eqn:system}) and (\ref{eqn:system2}) is expressed for the full sample as

\begin{align}
	\mathbf{y}_i = \mathbf{H}_i \boldsymbol{\gamma}_i + \mathbf{u}_i \label{eqn:matrix_system}
\end{align}

The OLS and GIVE estimators in this setup are:

\begin{align}
	\hat{\boldsymbol{\gamma}}_{i, OLS} &= (\mathbf{H}_i^{\prime} \mathbf{H}_i)^{-1} \mathbf{H}_i^{\prime} \mathbf{y}_i	\label{eqn:ols}\\	
	\hat{\boldsymbol{\gamma}}_{i, GIVE} &= (\mathbf{H}_i^{\prime} \mathbf{P}_{\mathbf{W}_i} \mathbf{H}_i)^{-1} \mathbf{H}_i^{\prime} \mathbf{P}_{\mathbf{W}_i} \mathbf{y}_i \label{eqn:give}	
\end{align}

with the theoretical feasible covariance matrices:

\begin{align*}
	\hat{\mathbf{V}}_{i, OLS} &= \hat{\sigma}^2_i (\mathbf{H}_i^{\prime} \mathbf{H}_i)^{-1} \\
	\hat{\mathbf{V}}_{i, GIVE} &= \hat{\sigma}^2_i (\mathbf{H}_i^{\prime} \mathbf{P}_{\mathbf{W}_i} \mathbf{H}_i)^{-1}
\end{align*}


where $\hat{\mathbf{u}}_i = \mathbf{y}_i - \mathbf{H}_i \hat{\boldsymbol{\gamma}}_i$ and $\hat{\sigma}^2_i = \frac{1}{T} (\hat{\mathbf{u}}_i^{\prime} \hat{\mathbf{u}}_i)$. 

%need to be precise here
If the parameter values $c_1, c_2$ are known, then the system is linear in parameters even though one of the regressors is a nonlinear function of the endogenous variables from other equations. \cite{pesaran2007econometric} and \cite{massacci2007identification} use the country specific exogenous regressors as instruments for the crisis dummy associated with that country which appears in the other equations. \cite{metiu2012sovereign}, using an alternative identification strategy to the one presented here, uses lagged terms of the dependent variables for the other countries in the system. 

Neither of these choices of instrument is ideal. The optimal choice of $w_{i,t}$ would be the conditional probability of a crisis occurring at time $t$, given the information set $\boldsymbol{\Omega}_t = (\mathbf{F}_t^\prime, \mathbf{x}_{i,t}^{\prime}, \mathbf{x}_{j_t}^{\prime})$. As this is a function of unknown parameters $\boldsymbol{\gamma}_i$, it is infeasible. The instruments chosen are likely to suffer from a weak instrument problem, in which case the distribution of the GIVE estimator will not be asymptotically normal, meaning standard statistical inference may be misleading. A \cite{cragg1993testing} statistic is reported to indicate whether this problem is severe, and refer to the critical values given in \cite{stock2005testing}.

\subsubsection{Full Information Maximum Likelihood}
\label{fiml}

\cite{massacci2007identification} takes the model in PP, and derives a Full Information Maximum Likelihood (FIML) estimator, treating the crisis dummies as missing data that are generated by latent variables which are a function of other elements of the dataset. Using Monte Carlo simulations, Massaccci finds that FIML performs better than GIVE, while imposing fewer requirements on the data by not requiring equation specific variables to identify the contagion coefficients. Instead, identification is achieved by exploiting the dichotomous outcomes of the indicator function. This means that the identification condition is that at least one observation of $s_{i,t}$ falls either side of the threshold variable $c_i$. 

Though this estimator could not be implemented for the system in question, a short exposition drawing on \cite{massacci2007identification} is presented for reference. 

First, the assumptions that the elements of $\Omega_t$ are ergodic stationary variables, and that the errors $\mathbf{u}_t~\sim~\text{IID}(\mathbf{0}, \mathbf{\Sigma_u})$ are imposed. 

Let $\boldsymbol{\theta}_k$ denote the vector of coefficients which characterise the joint distribution of $(s_{1,t}, s_{2,t})^{\prime}$ in regime $k$ for $k = 1, 2, 3, 4$ and the regimes correspond to those in (\ref{eqn:regions}):

\begin{align*}
	\label{eqn:fiml_coefs}
	&\boldsymbol{\theta}_1 \equiv (\boldsymbol{\phi}^{\prime}_1, \boldsymbol{\delta}^{\prime}_1,  \boldsymbol{\phi}^{\prime}_2, \boldsymbol{\delta}^{\prime}_2), 	&\boldsymbol{\theta}_2 \equiv ~~~ (\boldsymbol{\phi}^{\prime}_1, \boldsymbol{\delta}^{\prime}_1,
	\beta_1,  \boldsymbol{\phi}^{\prime}_2, \boldsymbol{\delta}^{\prime}_2) \\
	&\boldsymbol{\theta}_3 \equiv (\boldsymbol{\phi}^{\prime}_1, \boldsymbol{\delta}^{\prime}_1,  \boldsymbol{\phi}^{\prime}_2, \boldsymbol{\delta}^{\prime}_2, \beta_2), 	&\boldsymbol{\theta}_4 \equiv ~~~ (\boldsymbol{\phi}^{\prime}_1, \boldsymbol{\delta}^{\prime}_1,
	\beta_1,  \boldsymbol{\phi}^{\prime}_2, \boldsymbol{\delta}^{\prime}_2, \beta_2)
\end{align*}

Denoting the joint probability density function as $f(s_{1,t}, s_{2,t}; \boldsymbol{\theta_k} | \boldsymbol{\Omega}_t)$, then the system in (\ref{eqn:system}) and (\ref{eqn:system2}) can be represented:

\begin{align}	\label{eqn:fiml_decomp}
	f(s_{1,t}, s_{2,t} | \boldsymbol{\Omega}_t) =& [1 - \mathbf{I}(s_{1,t} - c_1)][1 - \mathbf{I}(s_{2,t - c_2})] f(s_{1,t}, s_{2,t}; \boldsymbol{\theta_1} | \boldsymbol{\Omega}_t)\\	
	&+ [1 - \mathbf{I}(s_{1,t} - c_1)]\mathbf{I}(s_{2,t - c_2})f(s_{1,t}, s_{2,t}; \boldsymbol{\theta_2} | \boldsymbol{\Omega}_t)\\
	&+ \mathbf{I}(s_{1,t} - c_1)[1 - \mathbf{I}(s_{2,t - c_2})]f(s_{1,t}, s_{2,t}; \boldsymbol{\theta_3} | \boldsymbol{\Omega}_t)\\
	&+ \mathbf{I}(s_{1,t} - c_1)\mathbf{I}(s_{2,t - c_2})f(s_{1,t}, s_{2,t}; \boldsymbol{\theta_4} | \boldsymbol{\Omega}_t) \label{eqn:fiml_decomp2}
\end{align}

\cite{massacci2007identification} shows that, thanks to the presence of multiple equilibria, the sum of the above generates a cdf that exceeds unity. He applies a normalisation factor $\frac{1}{p_t}$, defined such that the sum of probabilities is equal to one. 

Applying this, he uses the $f(s_{1,t}, s_{2,t} | \boldsymbol{\Omega}_t)$ decomposed in (\ref{eqn:fiml_decomp}) - (\ref{eqn:fiml_decomp2}) to maximise the following log-likelihood function:

\begin{align}
	L_T = \sum_{t=1}^{T} \text{log}f(s_{1,t}, s_{2,t} | \boldsymbol{\Omega}_t)
\end{align}

\cite{massacci2007identification} provides a proof of the consistency of parameter estimation using this estimator.

The estimator's major strength when estimating contagion coefficients is that it does not require country-specific explanatory variables in order to be identified. This is especially useful in the face of weak instruments which could render GIVE unable to provide reliable inference. It faces challenges regarding the assumptions it requires about the underlying distribution of the determinants of $D_{j,t}$, which in turn require consistent estimates of $c_i$ and $c_j$. 

Additionally, extending the estimation technique to the multiple country setting is challenging, and estimation of country-pairs does not lend itself to the task of studying contagious systems. As a result, there are no estimations performed using the estimator, though it would provide fruitful future work.

\subsection{Crisis Dating}
\label{dating_methodology}

A key concern in all the above methods described is the procedure used to determined $D_{j,t}$. There is little consensus on how to solve this problem. \cite{fry2011actually} survey the literature on five crises since the mid-1990s and find over seventy papers which only rarely agree on precise dating of trigger events. Figure 1 in  \cite{dungey2015endogenous} (DMTY) vividly illustrates the lack of consensus on dating the recent U.S. financial crisis.

Though small disagreements in the dating methodology may not seem a major concern, misspecification of crisis periods will almost certainly lead to inconsistent estimation of the contagion coefficients. If the mistakes are systematic, for example thanks to sample selection bias, then this can have serious implications for drawing meaningful inference from the model. 

The level of attention paid to this issue varies substantially in the literature. DMTY identify three  features commonly used by scholars to date crises: increases in volatility above a threshold; changes in transmission mechanisms between countries; and institutional action.

Institutional action, such as a devaluation, bailout, policy rate change, is the most common approach in early studies. This is unsurprising, as is it most closely linked to the popular understanding of crises\footnote{For instance, discussing the Asian financial crisis, a contemporary BBC article states clearly that ``It was the devaluation of the Thai baht in July that began a chain of currency devaluations across the region." This is representative of the mainstream popular view.
	\textit{Source:} \texttt{http://news.bbc.co.uk/1/hi/special\_report/1997/asian\_economic\_woes/34487.stm}}. However, policy action is almost certainly endogenous to underlying crisis variables, as policy often responds with a lag to developments in the macroeconomy or currency/bond markets rather than causing those developments. As it is the evolution of the underlying variable that is of interest, using such a strategy will likely misspecify crises periods. This is true both of the start and end of episodes. 

Inter-country transmission mechanisms are more valid in terms of econometric theory, but do not remain stable over time even when there is no crisis. Innovations or underlying structural changes to cross-border relationships can confound attempts to use the stability of these linkages as an indicator of contagion. Over longer periods of time, this requires special attention and modelling. However, sudden shifts in the mechanisms, such as those we may observe over short time-spans, hint at a move between equilibria rather than long term trends. This can be used to help with crisis dating.

Finally, volatility tends to increase sharply in crisis periods, and can almost be seen as the definition of crisis. A range of papers (notably \cite{eichengreen1996contagious} and \cite{pesaran2007econometric}) seeking to avoid the sample-selection bias inherent in more subjective approaches, use an indicator function that triggers when the performance measure rises above a certain multiple of its standard deviation. This approach attempts to allow the data to determine crisis periods, and reduce researcher discretion in the selection process.

DMTY combine smooth transition function framework with a structural GARCH model to exploit the latter two factors and endogenously determine the crisis dates. They simultaneously estimate the contagion coefficients using a VAR structure. \cite{metiu2012sovereign} uses a one-step-ahead Value-at-Risk (VaR) model to denote periods of stress, while \cite{massacci2007identification} utilises a grid search process. A final notable example of an attempt to solve the identification problem is found in \cite{brutti2012transmission}, where the authors employ the `narrative approach' of \cite{romer1989does} to select days on which news specifically relating to Greece was broken during 2009-10. By stripping out days on which potentially confounding news about other countries broke, contagion originating solely in Greece can be identified and the effect of financial linkages, the initial research question, can be estimated. \cite{arezki2011sovereign} apply a similar approach, using sovereign credit rating news as the narrative indicator. 

The above serves as a demonstration that there is relativity little consensus among scholars as to the optimum strategy for specifying crisis periods. The two most common, and also simplest to implement, are a pseudo narrative approach, where key events are selected as start- and end-points for the crisis, and a data-driven indicator function method. These are the two approaches used in this paper, with the details described below.

\underline{Date Scheme A}

As institutional action is such a popular indicator in the literature, a form of narrative dating approach is presented first. Four Eurozone countries, Greece, Ireland, Spain, and Portugal, have received official bailouts since 2009. Table \ref{tab:bailout_dates} outlines the dates chosen as the start and end dates of each of the associated crises, along with sources and notes on selection. These notes further emphasise the arbitrary nature of subjectively selecting crisis dates after the fact. For instance, should a bailout crisis be declared over at the reaching of an agreement with international creditors, or after that agreement passes through the national parliament? While elements of this issue could be resolved with a more systematic narrative approach, the subjective compilation of timelines - usually by newspapers - required for such an exercise is likely to still introduce selection bias. 

The a priori specification of the crisis periods means that the dummies can theoretically be treated as predetermined. However, given that narratives about the Eurozone crisis revolve around bond market behaviour, endogeneity is still likely to be a problem - even if the crisis periods are well specified.

\underline{Date Scheme B}

The second method exploits the heightened volatility evident in crisis times, and uses the indicator function given in \cite{pesaran2007econometric}:

\begin{align}
	D_{j, t} \equiv \mathbf{I}(y_{j,t} - \kappa \sigma^2_{j})
\end{align}

In Section 6, PP estimate the \cite{eichengreen1996contagious} dataset using $\kappa = 2$. It is worth noting that the use of the unconditional standard deviation can again introduce sample selection bias to the procedure. For instance, failure to include previous crises in the sample period will lower the estimated unconditional standard deviation and thus increase the frequency of $D_{j,t} = 1$ in future crises. Conversely, inclusion of much older crisis data may generate $D_{j,t} = 0$ observations even when the realisations of $y_{j,t}$ are abnormally high when considering only the contemporary period. 

In this scheme, the PP value of $\kappa_j = 2$ is retained, with the standard deviation calculated over the whole sample period. An alternative scheme, using the \cite{eichengreen1996contagious} value of $\kappa = 1.5$ is retained for robustness checks.


\end{document}
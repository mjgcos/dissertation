\documentclass[/../base.tex]{subfiles}

\begin{document}

\begin{center}
%\includegraphics[width=0.6\textwidth]{logo.jpg}~\\[1.5cm]
%\textsc{\Large }\\[0.5cm]
\rule{1\textwidth}{2pt}\\
{ \LARGE \bfseries Identifying Contagion as a Cause of The Euro Crisis  \\[0.2cm] }
\rule{1\textwidth}{2pt}\\[2cm]

\noindent
\begin{minipage}[t]{0.45\textwidth}
\begin{flushleft} \large
\emph{Author:}\\
\textsc{B064931}\\
\textbf{}
\end{flushleft}
\end{minipage}%
\begin{minipage}[t]{0.4\textwidth}
\begin{flushright} \large
\emph{Supervisor:} \\
Dr.~Andy \textsc{Snell}\\
\textbf{The University of Edinburgh}

\end{flushright}
\end{minipage}\\

\vfill


    \begin{abstract}
		 The study of contagion has become a key element of modelling financial crises. Using a canonical approach developed by Pesaran and Pick (2007), this paper describes the theoretical underpinnings of multiple equilibria contagion models, and presents two distinct estimation strategies. Problems of endogeneity and weak instruments are discussed, and estimation using Generalised Instrumental Varialbes yields results that are not consistent with the literature. Possible causes of insignificance are examined, with weak instruments and the presence of a unit root contributing.  
    \end{abstract}

\end{center}
\vfill

\noindent
\textbf{JEL Classification:} C10; G10; G15 \\
\textbf{Keywords:} Eurozone; Contagion; Identification; Multiple Equilibria; Financial crises




\end{document}
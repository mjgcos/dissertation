\documentclass[/../base.tex]{subfiles}
\begin{document}
\section{Introduction}
\label{intro}

Following swiftly on the heels of the global financial crisis, the revelation in October 2009 that Greece's deficit and sovereign debt burden were far larger than previously reported marked the beginning of a crisis  that has come close to unravelling the Eurozone. This is surprising for a number of reasons. In Europe, Greece is a peripheral state, both geographically and in economic terms, accounting for just 2.5\% of Eurozone output in 2008\footnote{Source: Eurostat \texttt{http://appsso.eurostat.ec.europa.eu/nui/show.do?dataset=nama\_10\_gdp\&lang=en}}. Unlike, for example, France and Germany, it does not host financial institutions of systemic importance for credit supply to other Eurozone economies. Greece also has a turbulent financial history, with \cite{kalyvas2015modern} reporting that the state has spent almost half of its independent history in default.

It is therefore an intriguing research puzzle to ascertain how the news that a small country, seemingly structurally irrelevant and with a history of sovereign default, was in financial difficulty could have such profound effects. The international crisis that followed the announcement has left a group of stable, developed countries in a prolonged slump, with stagnant growth and double-digit unemployment rates. The question of whether Greece was a cause of, or merely a trigger for the events that followed is of crucial importance for the continuation of the Euro project, as well as academic understanding of macroeconomic financial linkages.

\cite{reinhart2008time} identify three avenues through which debt sustainability issues are normally resolved: inflation, financial repression, and default. Surrendering control of monetary policy and membership of the common market render the former pair legally difficult, leaving default as the prime candidate for releasing pressure on government finances. Nevertheless, market and policy maker reaction to the news was panic, the implications of which are still being played out today.

This surprise was shared by the academic community. \cite{whelan2013sovereign} reviews the debate surrounding the adoption of the Euro, and finds relatively little attention paid to the prospect of sovereign default. The aftermath of the crisis has generated a huge amount of output, particularly regarding the mechanisms by which shocks can be transmitted across markets. 

While obvious linkages such as trade and capital flows are vitally important when modelling such a system, an increasing quantity of research has focused on the concept of contagion as an important determinant in the propagation of crises. Though formally defined slightly differently in Section \ref{lit}, contagion is usually portrayed as investor confidence in one country decreasing as a result of a crisis elsewhere, and thus creating the conditions for a downturn in the former. The specific focus of this study, given the above motivation, is the conditions under which contagion can occur in the market for sovereign debt.

Bond yields are the interest rate paid by governments on sovereign debt. As such, they can be seen as the market's expectation of the state's ability to repay, which itself is a function of the underlying performance of the economy. The first order determinants of bond spreads will therefore be differences between domestic macroeconomic conditions (both in terms of observable variables, like GDP growth, and unobserved institutional structures) and the foreign country. However, in open economies, it is clear that international events will play a role. For example, an adverse shock to a trading partner may lead to a domestic currency appreciation, reducing net exports and having implications for factors such as the current account balance and domestic GDP. Contagion effects, it is supposed, make up the balance.

Accurately modelling the evolution of bond spreads during the Euro crisis is of significant interest to policy makers. Figure \ref{fig:lit_gpiigs} demonstrates the effect both of Euro adoption, where yields converged to the German level, and the crisis. Since 2008, Europe has been clearly divided between the `core', where yields have declined from their pre-crisis level to nearly zero, and the `periphery', which has experienced an abrupt return to high rates. In more recent years, peripheral countries have seen declines in borrowing costs, but the effects of their abrupt ejection from debt markets are still impacting the real economy. Clearly it is of interest what proportion of that shock was caused by changes in investor beliefs - which might be mitigated by policy action or institutional design - or through more traditional channels such as trade.

\begin{figure}
	\centering
	\includegraphics[scale = 0.65]{../img/lti_gpiigs2.pdf}
	\caption{Long term bond yields for Germany and the PIIGS nations. Euro adoption date and Greek debt announcement denoted by vertical lines. \textit{Source:} European Central Bank}
	\label{fig:lit_gpiigs}
\end{figure}

A plausible cause of contagion in the Euro crisis is the institutional design of the Euro itself, as well as its relatively new and untested structures.  \cite{obstfeld1997destabilizing} warns that currency regimes lose many of their appealing characteristics when get-out clauses are included; at some point, it will likely become optimal for a country to leave, which can itself bring about fears from investors that it might. As agents learned what governments (and the ECB) were willing to do to keep the currency union together, there may have been a relaxing in attitudes towards the periphery. If this is the case, then an eventual Greek exit could spark another devastating burst of contagion. It is likely the desire to mitigate this that has motivated European leaders to prevent an exit thus far, and it also motivates this study.

The rest of paper continues as follows. Section \ref{lit} surveys the literature and major empirical results. Section \ref{est} details the econometric issues surrounding the identification of contagion, and outlines the estimation strategy. Section \ref{data} describes the dataset and some key characteristics, while section \ref{results} analyses the results. Section \ref{conclusion} concludes.



\end{document}
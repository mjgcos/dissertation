\documentclass[/../base.tex]{subfiles}
\begin{document}
\section{Introduction}
\label{intro}

Greece a tiny, peripheral (economic and geographic) European outpost that has been in some form of default for almost half of its independent history (\cite{kalyvas2015modern}). How did revelation that it had unsustainable debt cause (directly or indirectly) a crisis of sufficient magnitude to force countries as removed as Ireland, Spain, and Portugal to the brink of default, nearly collapsing the entire Euro project in the process? 

Seeking to explain evolution of bond spreads in Eurozone since 2007.

\cite{whelan2013sovereign}

Big policy implications. The crisis locked member states out of sovereign bond markets and caused incredible instability to the point where the viability of the currency union was severely questioned. Response to crisis has been hugely controversial, with taxpayer funded bailouts, depositor bail-ins, wide-ranging and untested policy commitments (e.g. OMT) etc. Political instability stemming partly from perceived encroachment on democratic institutions a long term problem and Greek situation still unresolved. Still widespread recession and low growth in Europe.  

While the crisis was in full swing sober policy analysis was harder to accomplish. Notwithstanding the recent showdown in Greece, the abatement of the worst excesses of the crisis has allowed academic research on the period to flourish. This analysis is shedding new light on some of the implications of the policy decisions taken in the midst of various episodes of market panic. An infamous example of hindsight shedding light on a disastrous decision is the Irish bank guarantee of 2008 - a move that essentially bankrupted the State and resulted in a bailout by the European Commission, ECB, and IMF. The decisions of this so called Troika have also come under scrutiny, notably in the form of recent internal IMF research suggesting Greek debt was unsustainable without hefty restructuring, and that the Fund erred by providing funds to an insolvent state. 

These examples, along with many others, demonstrate the importance of understanding exactly what happened in the Eurozone after 2009. As the catalyst for many of the worst episodes of the crisis was loss of access to bond markets, it is intuitive to focus on modelling the behaviour of sovereign bond yields when analysing the transmission of shocks through the Euro area. 

Bond yields are the interest rate paid by government on sovereign debt. As such, they can be seen as the market's expectation of the state's ability to repay, which is itself a function of the underlying performance of the economy. The first order determinants of bond spreads will therefore be differences between domestic macroeconomic conditions (both in terms of observable variables like GDP growth, and unobserved institutional structures) and the foreign country. However, in open economies, it is clear that international events will play a role. For example, an adverse shock to a trading partner may lead to a domestic currency appreciation, reducing net exports and having implications for factors such as the current account balance and domestic GDP. It is these international spillovers that form the primary concern of this study. 

Particular notes on contagion

Rest of paper continues as follows. Section \ref{lit} surveys the theoretical literature and major empirical results. Section \ref{est} details the econometric issues surrounding the study of contagion, and outlines the estimation strategy. Section \ref{data} describes the dataset and some key characteristics, while section \ref{results} analyses the results and section \ref{conclusion} concludes.


\begin{figure}
	\centering
	\includegraphics[scale = 0.65]{../img/lti_gpiigs2.pdf}
	\caption{Long term bond yields for Germany and the PIIGS nations. Euro adoption date and Greek debt announcement denoted by vertical lines. \textit{Source:} European Central Bank}
	\label{fig:lit_gpiigs}
\end{figure}


\end{document}
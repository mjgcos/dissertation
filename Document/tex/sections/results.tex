\documentclass[/../base.tex]{subfiles}

\begin{document}
\section{Results}
\label{results}

Results for the contagion coefficient estimates are recorded in Appendix \ref{a:tables}, while further output is contained in the text files included with the submission. The reported values come from the model with $\kappa = 1.5$, but there are no important differences from the $\kappa = 2$ specification. The estimations using Dating Scheme A does not provide any contagion coefficients that are statistically significant.

The OLS results demonstrate a surprising pattern, with the majority of significant contagion coefficients occuring in the `core' class of countries. Despite being biased and inconsistent, the OLS estimates match up reasonably close to \cite{metiu2012sovereign}, albeit the results in \ref{a:tables} are slightly smaller - and even negative in the case of France. These anomalies are anticipated by the simultaneous nature of the crisis dummies.

The more surprising results are listed in Table \ref{tab:contagion_give_a}. To take the GIVE results at face value, it would appear that there is no evidence of contagion during the Euro crisis. Only one coefficient, the Irish dummy in the equation for Portugal, is statistically distinguishable from zero at the 5\% level, and even this effect vanishes in the stricter $\kappa = 2$ setting ($p = 10.925$). The estimation could therefore be taken as evidence to support the \cite{forbes2002no} result of `no contagion, only interdependence' in the context of the Eurozone. It is also in stark contrast with \cite{metiu2012sovereign}, who finds significant contagion coefficients in nearly every equation, with Greece having an especially important impact. 

However, there are considerable reasons to doubt the validity of this conclusion. Firstly, the dating strategy allows the data to determine the crisis points, but that does not prevent endogeneity problems arising. Most obviously, there is likely to be omitted variable bias, where some common factor is forcing both stock market performance and bond yields. Simultaneity may also act through investor behaviour. Performance in one market may generate a reaction in the other as agents rebalance their portfolios.   

Additionally, there are aspects of the data analysis that may have contributed to such markedly different results to those commonly found in the literature. Late in the testing process, it became apparent that the version of the HAC variance-covariance estimator used to estimate the OLS results was not interacting with the R package AER, from where the instrumental variables function is sourced. It may therefore be the case that the evidently heteroskedastic data generating process (see Figure \ref{fig:dbsp}) may have resulted in wider confidence intervals, leading to the rejection of the null in more often than with robust standard errors. An autocorrelation robust estimator was operable, but made no notable difference to the standard errors calculated. This is an important limitation of the results, and could be resolved by using a routine such as Stata's \texttt{ivreg2}.  

Crucially, the dependent variable used in this process is not the same as the level of bond spreads used in most empirical applications. When the analysis was performed using the undifferenced spreads, large and highly significant contagion coefficients were found. This specification was obviously incorrect; due to the presence of a unit root, the regression is spurious. \cite{metiu2012sovereign} does not mention the possibility of spreads being non-stationary, implying that it was not an issue with his data. This may well be a function of timing. As can be clearly seen in Figures \ref{fig:lit_gpiigs} and \ref{fig:dbsp}, Eurozone bonds showed essentially no persistent deviation from the German level until the financial crisis, after which peripheral bond yields resemble a random walk. Metiu's dataset, running from January 2006 to February 2012, means that far less data falls in the crisis period than in the series under examination here. This may explain the higher estimates he finds for the contagion coefficients, as they may be the only elements in his model picking up the effect of the new regime. 


Intuitively, given the definition of contagion established in previous sections, this lends support to the presence of multiple equilibria in the data generation process, and highlight further the difficulties of detecting the presence of discontinuities close to when they occur. A model testing for the presence of structural breaks in the style of \cite{bai1998estimating} may be more appropriate given these circumstances. In particular, unit root tests that permit endogenously determined break dates, such as that proposed in \cite{zivot1992further}, could be especially useful for examining whether this form of contagion has occurred. Indeed, much of the previous empirical work on contagion, developed with currency crisis events in mind, does not appear to be applicable to the Eurozone case. Prior definitions of contagion refer to the phenomenon as an increase in correlations during episodes, whereas the Euro crisis is characterised by \textit{lower} comovements between bond yields in the periphery. Similar findings have been made in the CDS market, as documented by \cite{caporin2013measuring}. 

Additional sources of error in the estimation procedure may be found in the presence of volatility clustering. As noted in DMTY, accounting for persistent volatility when determining crisis dates increases the precision of threshold-based dating mechanisms. Preliminary testing of the spread variables found evidence of GARCH effects in all series aside from the Netherlands, and incorporating this into the model in both the episode delimitation stage and the estimation itself would raise the power of the statistical tests on the contagion coefficients. 

A much bigger issue is the weakness of the instruments. \cite{cragg1993testing} establishes that an F-test of the first stage regression can indicate the strength of the instruments. Using the values tabulated in \cite{stock2005testing}, it should be possible to reject a null hypothesis of joint insignificance of the instrument in predicting each of the endogenous variables. Over the entirety of the three dating specifications, only one of these tests can be rejected at the 5\% level (the Netherlands dummy in the French equation - which we would not ex ante expect to be a source of any contagion).

This presents a significant problem for drawing inference from the model. As noted in \cite{massacci2007identification}, "[when instruments are weak] the GIVE estimator does not have an asymptotically normal distribution, and standard statistical inference provides misleading results."  The coefficient estimations are biased, and will not converge to the population values in probability limit. \cite{mondria2013financial} observe that the direction of this bias for the contagion estimates towards zero, giving more credence to the contrast with \cite{metiu2012sovereign}. With weak instruments, GIVE is inconsistent, and will not provide reliable evidence. 

This is evidence that can be used for a future research strategy. Section \ref{data} has already documented the issues surrounding country-specific data collection, but the model presented in \cite{pesaran2007econometric} relies heavily on having access to good instruments to identify the contagion dummies. The evidence presented here suggests that domestic stock indices are not well suited to this role - likely thanks to correlations with the common factors. The results could also lend support to the contention in \cite{massacci2007identification} that GIVE is a less than ideal estimator for contagion, and that alternatives should be sought, such as the one presented in Section \ref{fiml}.


\end{document}
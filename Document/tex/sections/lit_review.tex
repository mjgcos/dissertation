\documentclass[../base.tex]{subfiles}

\begin{document}
\section{Literature Review}



\cite{masson1998contagion} decomposes international linkages into three categories. Spillover effects, which act through channels such as trade and competitiveness, and monsoonal effects, which come from the global environment and affect all sample countries, account for the majority of interconnectivity between countries. The final category is the unexplained residual, caused by self-fulfilling beliefs about one country prompted by a crisis in another. This has become the standard definition of contagion in the literature, and is seen to have a crucial explanatory role in the analyses of international financial crises.

Theoretical models of cross-border crises have traditionally focussed on speculative attacks on fixed exchange rate regimes. Extensions of the model proposed in \cite{krugman1979model} have generally assumed a speculative attack, the signal for the economy to move from the `good' to `bad' state equilibrium, is triggered by a steady reduction in government reserves in the defence of an overvalued currency peg. At some non-zero level of reserves, the attack is launched and the peg abandoned. 

Empirics, further support. When assuming rational expectations, it seems reasonable to expect convergence to a unique equilibrium (Barro Gordon as example?)

The alternative to the deterministic convergence framework is a multiple-equilibria model. \cite{masson1999contagion} proposes an analytically solvable balance of payments model that generates multiple equilibria. The likelihood that reserves decline below a triggering threshold is influenced by the interest rate charged on external borrowings - which is in turn a function of expectations. As in \cite{jeanne1997currency} and \cite{obstfeld1996models}, for a given range of fundamentals, this will result in a non-unique solution. While these fundamentals include factors that reflect external conditions (such as world interest rates and the expected trade balance) to account for monsoonal or spillover effects, these models are essentially one country evaluation that do not account for sentiment towards other countries. 

Kiyotaki and Moore (2002) balance sheet contagion

Several major explanations: 
Kaminsky and Reinhart. Fast and furious conditions + contagion that never happened.

\cite{kaminsky2003unholy} explore two decades of major financial events, and attempt to evaluate why some crises had major international implications but the majority did not. They identify three conditions - an `unholy trinity' - required for a domestic crisis to become a `fast and furious' contagion episode. Such episodes are generally characterised by large inflows of capital that are subject to a `sudden stop' once the crisis has begun; they are unanticipated by financial markets; and they involve a leveraged common creditor 

A fundamental question at the heart of multiple equilibrium models is determining the channels through which expectations are altered enough to shift between states. \cite{goldstein1998asian} proposes a theory of `wake up call' contagion, whereby investors' signals about fundamental strength in one region are revealed to be too optimistic, causing a reassessment of conditions in a second region. \cite{ahnert2015wake} extend this framework to show that, in an information constrained setting where cross-border fundamental correlation is uncertain ex-ante, contagion can still occur even if no correlation is present and common lender effects are absent. 

It can be argued that this approach is, in fact, simply a single equilibrium model where transition to the long-run steady state is sped up by the event in the origin region. In a costly-information setting, however, optimal signal processing can be seen to be deeply affected by self-fulfilling market sentiment that can determine the perceived state of the world.

info allocation connected to WUC. Gorodnychenko, Simms, etc

herding (Bikhchandani Hirshleifer and Welch 1998) fad, informational cascades. Optimal for agent to follow previous agent regardless of own info. Banerjee 1992

How it comes about: most models find multiple equilibria are only possible in certain ranges for fundamentals, i.e. they are positive yet weak. Clear implication for policy makers is that if Euro crisis was primarily propagated through self-fulfilling changes in market sentiment, maintaining strong fundamentals is an important method of insulation. 


Obvious links to Greek situation, but was wake up moment sufficient?

Estimation method of Pesaran and Pick to account for multiple equilibria.


Many recent analyses of market reactions to sovereign default risk have used credit default swap (CDS) spreads to proxy for market expectations. It is important to note that, while ostensibly determined by a similar underlying process, the evolution of CDS premia and bond yields is not identical. As outlined in  (Fontana and Schiecher), the former measure suffers from several complications relating to investor risk-appetite and market liquidity that make it less suitable for drawing policy related conclusions.

Links with banking sector \cite{alter2012credit}



\cite{obstfeld1997destabilizing} - essentially predicts entire Euro crisis and talks about the negative impacts of even floating idea of Grexit.





















Empirical review. Masson 1998 puts 1980s debt crisis as monsoonal in response to US rate rise, EMS as spillover crisis, and Asian as not being satisfactorally explained by monsoonal or spillover. Link to Eichengreen 1996?






\subsection{Pesaran and Pick methodology}
Following Masson (\cite{masson1998contagion} and \cite{masson1999contagion}), three definitions of contagion. Monsoonal (global shocks), and spillover contagion are defined as `interdependence'; fundamental linkages that may or may not be observable account for the majority of the contagious comovements, which should in theory be predictable from macro fundamentals. Pure contagion, by contrast, is defined as a jump between equilibria (usually a `good' and `bad' state of the world) that is not forecastable through fundamentals alone. 

Taking Masson's definitions as standard requires a theoretical model that allows for all three methods of shock transmission between countries. Appropriate estimation of this model will 



\end{document}
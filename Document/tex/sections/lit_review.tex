\documentclass[../base.tex]{subfiles}

\begin{document}
\section{Literature Review}
\label{lit}

\cite{masson1998contagion} decomposes international linkages into three categories. Spillover effects, which act through channels such as trade and competitiveness, and monsoonal effects, which come from the global environment and affect all sample countries, account for the majority of interconnectivity between countries. The final category is the unexplained residual, caused by self-fulfilling beliefs about one country prompted by a crisis in another. This has become the standard definition of contagion in the literature, replacing a more general definition of contagion as an increase in cross-country correlations during crisis times. It is seen to have a crucial explanatory role in the analyses of international financial crises. 
%Simply form of parameter stability test, could be caused by anything including pre-crisis sampling error?

Theoretical models of cross-border crises have traditionally focussed on speculative attacks on fixed exchange rate regimes. Extensions of the model proposed in \cite{krugman1979model} have generally assumed that a speculative attack, the signal for the economy to move from the `good' to `bad' state equilibrium, is triggered by a steady reduction in government reserves in the defence of an overvalued currency peg. At some non-zero level of reserves, the attack is launched and the peg abandoned.

Much like the influential model of \cite{barro1983rules}, speculative attack models with exogenous initial conditions will converge to unique equilibria under rational expectations. If the conditions are poor - for instance an overvalued currency peg - then the long-run equilibrium will be worse than in countries with favourable initial conditions. The attack merely accelerates the convergence process, rather than changing the long-run equilibrium.

%Empirics, further support.

The alternative to the deterministic convergence framework is allowing for multiple equilibria. Referring to states' access to capital markets, \cite{calvo1995varieties} summarises the intuition behind such an outcome: ``If investors deem you unworthy, no funds will be forthcoming and, thus, unworthy you will be." He further notes that ``[d]espite the appeal of this view in the popular press, however, this point of view has not made a significant dent into the profession's conventional wisdom (which is still dominated by the Krugman model) yet."

This is no longer the case. \cite{masson1999contagion} proposes an analytically solvable balance of payments model that generates multiple equilibria. The likelihood that reserves decline below a triggering threshold is influenced by the interest rate charged on external borrowings - which is in turn a function of expectations. As in \cite{obstfeld1996models} and \cite{jeanne1997currency}, for a given range of fundamentals this will result in a non-unique solution. While these fundamentals include factors that reflect external conditions (such as world interest rates and the expected trade balance) to account for monsoonal or spillover effects, these models are essentially single-market evaluations that do not account for sentiment towards other countries. 

There are several well known theoretical examples of shocks spreading through sectors or countries that are not fundamentally connected to each other. \cite{kiyotaki2002balance} demonstrate that balance sheet effects can mean negative shocks to a subset of leveraged, credit constrained firms can spread to all such entities if they are holding similar classes of assets.  \cite{kiyotaki1997credit} examine the direct propagation of shocks through credit chains, noting that institutions for whom constraints do not bind will be unaffected by contagion. Similar themes are explored in \cite{rochet1996interbank}. The parallels between this conclusion and the observed spread of sovereign debt panic throughout the peripheral Eurozone countries, without a similar effect on the core, are clear.

While multiple equilibria models allowing for self-fulfilling panics provide an initial framework for analysing crises, scholars have increasingly turned attention to the specific circumstances required to alter the prevailing equilibrium. \cite{kaminsky2003unholy} explore two decades of major financial events, and attempt to evaluate why some crises had major international implications but the majority did not. They identify three conditions - an `unholy trinity' - required for a domestic crisis to become a `fast and furious' contagion episode. Panics are characterised by large inflows of capital that are subject to a `sudden stop' once the crisis has begun; they are unanticipated by financial markets; and they involve leveraged common creditors such as banks or other financial institutions. The authors contend that absence of these elements will generally see the effects of a shock contained domestically, while their combined presence will facilitate a devastating international feedback loop. 

%need more stuff here. Perhaps some empirical bits about wether this has been found to be true?


%role of expectations.
A fundamental question at the heart of multiple equilibria models is determining the channels through which expectations are altered sufficiently to precipitate between states. \cite{goldstein1998asian} proposes a theory of `wake up call' contagion, whereby investors' signals about fundamental strength in one region are revealed to be too optimistic, causing a reassessment of conditions in a second region. \cite{ahnert2015wake} extend this framework to show that, in an information constrained setting where cross-border fundamental correlation is uncertain ex-ante, contagion can still occur even if no correlation is present and common lender effects are absent. 

It can be argued that this approach is, in fact, simply a single equilibrium model where transition to the long-run steady state is sped up by the event in the origin region. In a costly-information setting with private signals, however, optimal signal processing can be seen to be deeply affected by self-fulfilling market sentiment that can determine the perceived state of the world.

The Wake Up Call hypothesis is closely associated with models of rational inattention, a branch of theoretical literature attempting to model intertia in agents' reaction to new information.  \cite{coibion2010information} survey the literature and find evidence to reject a null of full-information rational expectations (FIRE) when comparing ex ante forecast revisions to ex post forecast errors. 

The authors motivate their empirical specification from two branches of rational expectations information friction models. \cite{NBERw8290} model agents who face fixed costs to update their information set. At the point of updating, the agents posses full information about fundamentals. Christopher Sims presents an alternative in a series of papers. \cite{sims1998stickiness}, \cite{sims2003implications}, and \cite{sims2006rational} argue that a more intuitive explanation is that agents cannot process all the information they receive. Even if collection of information is costless, processing is not once agents reach a capacity constraint. If the constraint binds, agents will rationally fail to attend to all available information. 

\cite{mondria2013financial} provide an empirical application of the attention allocation phenomenon to financial transmission mechanisms. Using the frequency of articles about specific Asian countries in the \textit{Financial Times} as a proxy for attention, the authors construct a model of investor attention allocation between markets. In equilibrium, investors will process information at equal rates, but if one region falls into crisis then it is optimal to process a greater proportion of news from the crisis market. The associated decline in attention to the non-crisis market induces a rise in volatility, lowering the expected price in this region. This channel of contagion holds even if fundamentals are uncorrelated. They find support for all except the final prediction, though weak instrumentation and the use of realised prices to proxy for expectations may bias the results towards zero. 

A further contribution from behavioural modelling is herd behaviour and the `madness of crowds' phenomenon. \cite{bikhchandani1998learning} builds on \cite{bikhchandani1992theory} and \cite{banerjee1992simple} with a model of fads and informational cascades. The authors show it can be optimal for players to follow the previous agent's action regardless of their own information. Further, groups will not necessarily converge on the true values if only actions are observable, rather than the underlying signals. This mechanism could plausibly contribute to explanations of contagion, through the  \cite{kaminsky2003unholy} `sudden stop' observation. Having followed others into certain markets, investors stampede for the exists in unison when the true fundamentals are revealed.

How it comes about: most models find multiple equilibria are only possible in certain ranges for fundamentals, i.e. they are positive yet weak. Clear implication for policy makers is that if Euro crisis was primarily propagated through self-fulfilling changes in market sentiment, maintaining strong fundamentals is an important method of insulation. 


Obvious links to Greek situation, but was wake up moment sufficient?

Estimation method of Pesaran and Pick to account for multiple equilibria.

Links with banking sector \cite{alter2012credit} and \cite{kaminsky1999twin}
Kaminsky: find that banking sector issues typically precede currency crises.


\cite{obstfeld1997destabilizing} - essentially predicts entire Euro crisis and talks about the negative impacts of even floating idea of Grexit.

\subsection{Empirical Studies of Contagion}

While a lively debate around the theoretical modelling of contagious crises has existed for several decades, serious attempts to estimate such models have only become commonplace in the aftermath of the Asian financial crisis of the late 1990s. One of the first such studies, \cite{eichengreen1996contagious}, finds considerable evidence that crisis in one country increases the probability of a contemporaneous crisis elsewhere. Noting that simply using devaluations would induce sample selection bias by excluding unsuccessful attacks, the authors use a crisis dummy that is triggered when a synthetic measure of Exchange Market Pressure moves more that 1.5 standard deviations above the mean. They also find evidence that trade linkages are the primary channels through which this form of contagion spreads. In a similar vein, \cite{esquivel1998explaining} attempt to use a structural model to forecast future (successful) attacks with a given information set.

A major issue common to many early studies is the frequency of the data used. As discussed in Section \ref{data}, lower frequency data can cause major issues with the identification of shocks and their effects. Access to higher frequency observations has improved in recent years, but there remain serious limitations on the use of structural macroeconomic models of contagion imposed by data availability for variables such as GDP and trade flows. Nevertheless, it is now possible to do considerably better than models involving quarterly, or even annual in the case of \cite{esquivel1998explaining}, data, which can require ad hoc - and tenuous - assumptions about causal linkages.

\cite{forbes2002no} demonstrate that, in the presence of positive and constant interdependencies, the estimated correlation between two markets will increase if one experiences a rise in volatility. As relatively high variance in macroeconomic indicators is a feature of crisis periods, they argue that interdependence was entirely responsible for the spread of crises at the end of the last century, and that contagion becomes insignificant when heteroskedasticity is accounted for. \cite{corsetti2005some} critique this approach by noting that the assumptions required to generate the `no contagion, only interdependence' result are too strong, and the test statistics used to measure contagion are biased downwards. The authors find that distinguishing between common-factor and country-specific components of the outcome variable results in far fewer rejections of a null of no structural break. 

A group of papers utilise more sophisticated time series techniques to estimate contagion. \cite{sander2003contagion} attempt to model the evolution of a system of sovereign bond spreads. They divide the Asian financial crisis into four phases, and estimate a series of bivariate VAR models before performing Granger causality analysis. They note that this is only valid in cases where both series are I(0), as is the case in only a small number of the relationships. Series that are I(1) are differenced to allow valid inference. Additionally, pairs that are cointegrated are augmented with an error correction term. They find significant differences in both short term dynamics and long run relations during crisis periods, though their results require predetermination of these dates. The authors justify using short time spans to estimate long term relationships by using higher frequency data than similar studies, although Monte Carlo results in \cite{pierse1995temporal} suggest this does not improve the power of their tests.

In a similar vein,  \cite{kalbaska2012eurozone} conduct Granger causality tests on a VAR estimation of daily CDS spreads in the period around the Greek bailout. Using Granger-causality tests, they find significant increases in cross-country interdependence during the crisis period, while Generalised Impulse Response Function analysis revealed Spanish and Irish shocks to have the largest impacts on the system. However, these responses were found to depend heavily on the number of lags used in the VAR model, and make use of strong crisis identification assumptions.

\cite{beirne2013pricing} estimate a monthly panel model of CDS spreads for 31 countries during the Euro crisis, and find evidence of herding contagion concentrated in a few European markets. They also find that fundamentals are a poor predictor of CDS spreads in the pre-crisis period, lending support to a Wake Up Call channel.

\cite{caporin2013measuring} model CDS and bond premiums for major Eurozone countries, using a variety of econometric techniques. [Read this one a bit more]

\end{document}